\documentclass{article}
\usepackage{amsmath}
\usepackage{tikz}
\begin{document}

\title{Electricity and Magnetism - Lecture 7 Notes}
\author{Joshua Clement}
\maketitle

\section*{Electric Field Due to a Parallel Plate Capacitor}

\begin{figure}[h!]
    \centering
    \begin{tikzpicture}[scale=1.5]
        % Define styles for the arrows
        \tikzset{
            black arrow/.style={->, thick, black}
        }

        % Draw the two parallel plates
        \draw[thick, red] (2, -3) -- (2, 3);
        \draw[thick, blue] (4, -3) -- (4, 3);

        % Draw electric field arrows between the plates
        \foreach \y in {-2.5, -2, -1.5, -1, -0.5, 0, 0.5, 1, 1.5, 2, 2.5} {
            \draw[black arrow] (2.2, \y) -- (3.8, \y);
        }

        % Labels for the electric fields
        \node at (5, 0) {$E = 0$};
        \node at (1, 0) {$E = 0$};
        \node at (3, 3.5) {$E = \frac{\sigma}{\epsilon_0}$};

    \end{tikzpicture}
    \caption{Electric field between two infinite parallel plates. Fields inside add up, while fields outside cancel.}
    \label{fig:parallel_plates_field}
\end{figure}

\begin{itemize}
    \item \textbf{Two Oppositely Charged Plates}: Create a nearly uniform electric field between them.
    \item \textbf{Field Between Plates}:
    \[
    E = \frac{\sigma}{\epsilon_0}
    \]
    where \(\sigma\) is the surface charge density and \(\epsilon_0\) is the permittivity of free space.
    \item \textbf{Fringe Field}: Charges on the outer edges of plates create a non-uniform field at the edges.
\end{itemize}

\section*{Electric Field of a Uniformly Charged Shell (Recap)}
\begin{itemize}
    \item \textbf{Outside the Shell} (\(r > R\)): The electric field behaves as if all charge is concentrated at the center.
    \[
    E = \frac{1}{4\pi\epsilon_0} \frac{Q}{r^2}
    \]
    \item \textbf{Inside the Shell} (\(r < R\)): The electric field is zero due to complete cancellation.
    \[
    E = 0
    \]
\end{itemize}

\section*{Electric Field of a Uniformly Charged Sphere}
\begin{itemize}
    \item \textbf{Outside the Sphere} (\(r > R\)): The electric field is similar to that of a point charge.
    \[
    E = \frac{1}{4\pi\epsilon_0} \frac{Q}{|r|^2} \hat{r}
    \]
    \item \textbf{Inside the Sphere} (\(r < R\)): The electric field increases linearly with distance from the center.
    \[
    E = \frac{1}{4\pi\epsilon_0} \frac{Q|r|}{R^3} \hat{r}
    \]
    where \(R\) is the radius of the sphere and \(r\) is the distance from the center.
\end{itemize}

\section*{Two Infinite Planes with Opposite Charges}

\begin{figure}[h!]
    \centering
    \begin{tikzpicture}[scale=1.5]
        % Define styles for the plates
        \tikzset{
            plate/.style={thick},
            field arrow/.style={->, thick, black}
        }

        % Draw the first set of parallel plates
        \draw[plate, blue] (-3, -3) -- (-3, 3);
        \draw[plate, red] (-1, -3) -- (-1, 3);

        % Label surface charges for first set
        \node at (-3, -3.3) {$-\sigma$};
        \node at (-1, -3.3) {$\sigma$};

        % Draw electric field between plates (first set)
        \draw[field arrow] (-1.2, 0) -- (-2.8, 0) node[midway, above] {$E = \frac{\sigma}{\epsilon_0}$};

        % Label the electric field outside first set
        \node at (-4, 0) {$E = 0$};
        \node at (0, 0) {$E = 0$};

        % Draw the second set of parallel plates
        \draw[plate, blue] (1, -3) -- (1, 3);
        \draw[plate, red] (3, -3) -- (3, 3);

        % Label surface charges for second set
        \node at (1, -3.3) {$-\sigma$};
        \node at (3, -3.3) {$\sigma$};

        % Draw electric field between plates (second set)
        \draw[field arrow] (2.8, 0) -- (1.2, 0) node[midway, above] {$E = \frac{\sigma}{\epsilon_0}$};

        % Label the electric field outside second set
        \node at (4, 0) {$E = 0$};

    \end{tikzpicture}
    \caption{Electric field between two sets of parallel plates with opposite surface charge densities. Fields are zero outside the plates, while between the plates the field is \( E = \frac{\sigma}{\epsilon_0} \).}
    \label{fig:parallel_plates_field}
\end{figure}

\begin{itemize}
    \item \textbf{Superposition Principle}:
    \begin{itemize}
        \item Two infinite planes with surface charge densities \(\sigma\) and \(-\sigma\).
        \item \textbf{Field Between the Planes}:
        \[
        E = \frac{\sigma}{\epsilon_0}
        \]
        \item \textbf{Field Outside the Planes}: The fields cancel, resulting in \(E = 0\).
    \end{itemize}
\end{itemize}

\section*{Spherical Shell vs. Solid Sphere}
\begin{itemize}
    \item \textbf{Spherical Shell}:
    \begin{itemize}
        \item \textbf{Outside} (\(r > R\)): Field behaves as if all charge is at the center.
        \item \textbf{Inside} (\(r < R\)): Field is zero.
    \end{itemize}
    \item \textbf{Solid Sphere}:
    \begin{itemize}
        \item \textbf{Outside} (\(r > R\)): Field similar to a point charge.
        \item \textbf{Inside} (\(r < R\)): Field increases linearly with distance from the center.
    \end{itemize}
\end{itemize}

\section*{Building a Solid Sphere from Spherical Shells}
\begin{itemize}
    \item \textbf{Concept}: A solid sphere can be thought of as a series of concentric spherical shells.
    \item \textbf{Electric Field Calculation}:
    \begin{itemize}
        \item Use symmetry to determine that the field outside is the sum of all individual shell fields.
        \item Inside a uniformly charged conducting sphere, \(E = 0\) in equilibrium as charges move to the surface.
    \end{itemize}
\end{itemize}

\end{document}