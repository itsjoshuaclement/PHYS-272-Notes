\documentclass{article}
\usepackage{amsmath}
\begin{document}

\title{Electricity and Magnetism - Lecture 3 Notes}
\author{Joshua Clement}
\maketitle

\section*{Conservation of Charge}
\begin{itemize}
    \item \textbf{Electric charge} is quantized.
    \item SI unit: \textbf{Coulomb (C)}.
    \item \textbf{Conservation of Charge}: The net charge of a closed system never changes.
    \item Charges can be created or destroyed in \textbf{(+,-) pairs} (e.g., electron-positron annihilation).
    \item Charge transfer occurs via \textbf{contact} (e.g., triboelectric effect, rubbing a balloon on hair).
\end{itemize}

\section*{Triboelectric Effect}
\begin{itemize}
    \item Transfer of charge through \textbf{contact/friction}.
    \item Rubbing produces many points of contact, facilitating charge transfer.
    \item Mechanisms:
    \begin{itemize}
        \item Breaking large molecules or transferring ions.
        \item Transfer of \textbf{electrons}.
    \end{itemize}
    \item Determines whether an object becomes \textbf{positively or negatively charged}.
\end{itemize}

\section*{Electric Dipole}
\begin{itemize}
    \item Consists of two equal but opposite charges, \(+q\) and \(-q\), separated by distance \(s\).
    \item \textbf{Dipole Moment (\(\vec{p}\))}:
    \[
    \vec{p} = q \vec{s}
    \]
    \item The \textbf{dipole moment} is not the electric field; it is a different physical quantity.
\end{itemize}

\section*{Electric Field of a Dipole}
\begin{itemize}
    \item \textbf{Electric Field Along the Axis} (\(r \gg s\)):
    \[
    \vec{E}_{\text{axis}} = \frac{1}{4\pi\epsilon_0} \frac{2p}{r^3} \hat{p}
    \]
    \item \textbf{Electric Field Along the Bisecting Plane}:
    \[
    \vec{E}_{\perp} = -\frac{1}{4\pi\epsilon_0} \frac{p}{r^3} \hat{p}
    \]
\end{itemize}

\section*{Dipole in an External Uniform Electric Field}
\begin{itemize}
    \item \textbf{Forces} on \(+q\) and \(-q\) have the same magnitude but opposite direction.
    \item The dipole experiences a \textbf{torque}:
    \[
    \tau = \vec{p} \times \vec{E}_{\text{app}}
    \]
    \item \textbf{Equilibrium Position}: Dipole aligns with the external field, minimizing potential energy.
    \[
    U = -\vec{p} \cdot \vec{E}_{\text{app}}
    \]
\end{itemize}

\section*{Polarization of Atoms}
\begin{itemize}
    \item An \textbf{atom} becomes polarized when placed in an electric field, forming an \textbf{induced dipole}.
    \item \textbf{Polarization}: The dipole moment per atom or per molecule.
    \item The \textbf{polarizability (\(\alpha\))} is proportional to the strength of the applied electric field.
\end{itemize}

\section*{Interaction Between Neutral Atom and Point Charge}
\begin{itemize}
    \item A \textbf{neutral atom} becomes polarized when placed near a point charge, forming an \textbf{induced dipole}.
    \item The interaction is always \textbf{attractive}.
    \item The point charge is always on the \textbf{axis} of the induced dipole.
    \item The interaction strength is proportional to \(\frac{1}{r^5}\).
\end{itemize}

\end{document}