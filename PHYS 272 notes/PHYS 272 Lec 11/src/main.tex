\documentclass{article}
\usepackage{amsmath}
\usepackage{tikz}
\usepackage{xcolor}
\begin{document}

\title{Electricity and Magnetism - Lecture 11 Notes}
\author{Joshua Clement}
\maketitle

\section*{Key Topics}
\begin{itemize}
    \item \textbf{Electric Potential of a Conductor}
    \item \textbf{Sharp Point Effect}
    \item \textbf{Sources of Magnetic Field}
    \item \textbf{Magnetic Field due to Moving Charges}
    \item \textbf{Biot-Savart Law for a Point Charge}
    \item \textbf{Cross Products: Right-Hand Rule and Mathematical Representation}
\end{itemize}

\section*{Electric Potential of a Spherical Conductor with Net Charge \(Q\)}
\begin{itemize}
    \item \textbf{Electric Potential} \(V\) for a spherical conductor of radius \(R\) and charge \(Q\):
    \[
    V(r) =
    \begin{cases}
        \frac{1}{4\pi\epsilon_0} \frac{Q}{R}=ct & \text{if } r \leq R \\
        \frac{1}{4\pi\epsilon_0} \frac{Q}{r} & \text{if } r > R
    \end{cases}
    \]
    \item \textbf{Potential Inside Conductor}: Constant, equal to \(\frac{1}{4\pi\epsilon_0} \frac{Q}{R}=ct\)
\end{itemize}

\section*{Sharp Point Effect}
\begin{itemize}
    \item \textbf{Electric Field Enhancement}: The electric field is stronger at sharp points due to higher surface charge density.
    \item \textbf{Practical Example}: Lightning rods use this effect to direct lightning strikes to a sharp point, where the field is strongest.
\end{itemize}

\section*{Magnetic Field Key Concepts}
\begin{itemize}
    \item \textbf{Moving Charges Create a Magnetic Field}: A magnetic field is generated when charges are in motion.
    \item \textbf{Magnetic Dipole}:
    \begin{itemize}
        \item A current-carrying loop and a bar magnet are examples of magnetic dipoles.
        \item Even a single atom can be a magnetic dipole.
    \end{itemize}
    \item \textbf{Magnetic Field Units}:
    \begin{itemize}
        \item \textbf{Tesla (T)}: Standard unit for magnetic field.
        \item \textbf{Gauss (G)}: Another unit for magnetic field, where \(1 \, \text{G} = 10^{-4} \, \text{T}\).
    \end{itemize}
\end{itemize}

\section*{Compass Needle and Magnetic Field}
\begin{itemize}
    \item \textbf{Key Idea}: The needle of a compass aligns with the net magnetic field at its location, regardless of the source of the field.
    \item \textbf{Examples}:
    \begin{itemize}
        \item When isolated, the compass points to Earth's geomagnetic pole.
        \item When near a magnet or current-carrying wire, the needle deflects according to the net magnetic field.
    \end{itemize}
\end{itemize}

\section*{Biot-Savart Law}
\begin{itemize}
    \item \textbf{Key Idea}: The Biot-Savart Law calculates the magnetic field generated by a moving charge.
    \item \textbf{Formula}:
    \[
    \vec{B} = \frac{\mu_0}{4\pi} \frac{q \vec{v} \times \hat{r}}{r^2}
    \]
    where \(\mu_0\) is the vacuum permeability, \(q\) is the charge, \(\vec{v}\) is the velocity, and \(\hat{r}\) is the unit vector pointing from the charge to the observation point.
\end{itemize}

\section*{Right-Hand Rule for Cross Products}
\begin{itemize}
    \item \textbf{Right-Hand Rule}: Used to determine the direction of the magnetic field resulting from a moving charge.
    \item \textbf{Magnitude of Cross Product}:
    \[
    |\vec{A} \times \vec{B}| = AB \sin\theta
    \]
    where \(\theta\) is the angle between vectors \(\vec{A}\) and \(\vec{B}\).
\end{itemize}

\section*{Interesting Facts About Magnetic Fields}
\begin{itemize}
    \item \textbf{Earth's Magnetic Field}:
    \begin{itemize}
        \item Strength ranges from \(0.25 \, \text{G}\) to \(0.65 \, \text{G}\) (25 to 65 \(\mu\text{T}\)).
        \item Generated by electric currents in the Earth's molten iron and nickel core.
        \item North geomagnetic pole is actually the South pole of Earth's magnet.
    \end{itemize}
    \item \textbf{Magnetic Field Strengths}:
    \begin{itemize}
        \item Typical Nd magnet: \(1 \, \text{T}\).
        \item Strongest permanent magnets: \(4.5 \, \text{T}\).
        \item Neutron stars: Magnetic fields up to \(10^4 - 10^{11} \, \text{T}\).
    \end{itemize}
\end{itemize}

\end{document}