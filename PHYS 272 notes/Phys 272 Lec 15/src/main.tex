\documentclass{article}
\usepackage{amsmath}
\usepackage{amsfonts}

\begin{document}

\title{Lecture 15: Electric Circuits - Notes}
\author{Joshua Clement}
\maketitle

\section*{Transient Response in Circuits}
\begin{itemize}
    \item Before connection: No current flows, system in \textbf{equilibrium}.
    \item After connection: \textbf{Transient state} occurs where the electric field (\textbf{E-field}) disturbs the equilibrium.
    \item The \textbf{E-field} propagates at the \textbf{speed of light}, establishing the steady state within nanoseconds.
\end{itemize}

\section*{Resistors}
\begin{itemize}
    \item A \textbf{resistor} is a part of the circuit that opposes the flow of electrons.
    \item In steady state, current remains constant throughout the circuit, but \textbf{electric field (E-field)} differs based on material properties and geometry.
    \item \textbf{Energy Conservation} in a circuit follows \textbf{Kirchhoff’s Loop Rule}:
    \[
    \Delta V_1 + \Delta V_2 + \Delta V_3 + \cdots = 0
    \]
    - This is analogous to the principle that what goes up must come down.
\end{itemize}

\section*{Circuit with a Resistor}
\begin{itemize}
    \item Example 1: A narrow section of wire increases the \textbf{E-field} in that section.
    \item Example 2: A material with lower \textbf{mobility (u)} increases the \textbf{E-field} to maintain constant current.
\end{itemize}

\section*{Kirchhoff’s Voltage Loop Law (Energy Conservation)}
\begin{itemize}
    \item The total voltage drop across all elements in a closed loop equals zero.
    \item Voltage differences are related to \textbf{energy per unit charge}:
    \[
    \Delta V = \frac{\Delta U}{q}
    \]
\end{itemize}

\section*{Batteries}
\begin{itemize}
    \item \textbf{Electromotive force (emf)}: The battery's ability to maintain a potential difference between terminals.
    \item \textbf{emf} is the energy input per unit charge, measured in volts but can originate from chemical, nuclear, or gravitational energy.
\end{itemize}

\section*{Resistors in Series}
\begin{itemize}
    \item Increasing the length or cross-sectional area of a resistor affects the current.
    \item Doubling the length halves the current; doubling the cross-sectional area doubles the current.
\end{itemize}

\section*{Power in Circuits}
\begin{itemize}
    \item Power is the amount of \textbf{energy transferred} or \textbf{work done} per unit time.
    \item The \textbf{brightness} of a bulb is proportional to the \textbf{power dissipated}.
    \item Formula for power in a circuit:
    \[
    P = IV = enAuE \times E \times L
    \]
    \item \textbf{Units of Power}: Watts (W), where \( 1 \, \text{Watt} = 1 \, \text{Joule/sec} \).
\end{itemize}

\section*{Effect of Two Batteries in Series}
\begin{itemize}
    \item Two batteries in series double the \textbf{emf}, doubling the current and power.
    \item This increases the \textbf{brightness} of a light bulb.
\end{itemize}

\end{document}
