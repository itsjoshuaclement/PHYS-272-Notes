\documentclass{article}
\usepackage{amsmath}
\usepackage{amsfonts}

\begin{document}

\title{Chapter 18: Electric Circuits - Notes}
\author{Joshua Clement}
\maketitle

\section*{Magnetic Field of a Current Loop}
\begin{itemize}
    \item Produces a \textbf{magnetic dipole moment}.
    \item Similar principles apply to \textbf{atomic magnetic moments} in permanent magnets.
\end{itemize}

\section*{Permanent Magnets}
\begin{itemize}
    \item Result from alignment of \textbf{atomic dipole moments}.
    \item \textbf{Ferromagnetic materials} (e.g., iron, cobalt, nickel) exhibit permanent magnetism.
    \item \textbf{Domains} form within ferromagnetic materials, aligning in the presence of an external magnetic field.
    \item \textbf{Paramagnetic} materials have weak attractions; \textbf{diamagnetic} materials experience repulsion.
    \item Heating a ferromagnet can demagnetize it by disrupting domain alignment.
\end{itemize}

\section*{Key Concepts in Electric Circuits (Chapter 18)}
\begin{itemize}
    \item \textbf{Surface Charges} on wires create the \textbf{electric field (E-field)} that drives current in a circuit.
    \item \textbf{Steady State} follows initial \textbf{transient effects}.
    \item A \textbf{battery} maintains charge separation and potential difference.
    \item Circuit analysis involves two primary laws:
    \begin{enumerate}
        \item \textbf{Current Node Rule} (Kirchhoff’s Law): Current into a node equals current out.
        \item \textbf{Voltage Loop Rule}: Total potential difference around a loop is zero.
    \end{enumerate}
    \item \textbf{Conventional Current}:
    \begin{itemize}
        \item Electrons move opposite to conventional current flow.
        \item \textbf{Electron current} exits the negative terminal and enters the positive terminal.
    \end{itemize}
    \item \textbf{Drift Speed Formula}:
    \[
    I = q n A v
    \]
    where:
    \begin{itemize}
        \item \( I \): current
        \item \( n \): electron density
        \item \( A \): cross-sectional area
        \item \( v \): drift speed
    \end{itemize}
    \item \textbf{Equilibrium vs. Steady State}:
    \begin{itemize}
        \item \textbf{Equilibrium}: No current; electron drift velocity is zero.
        \item \textbf{Steady State}: Constant current; steady drift velocity.
    \end{itemize}
    \item \textbf{Energy Conversion in Circuits}:
    \begin{itemize}
        \item Current is not ``used up'' in a circuit.
        \item The \textbf{bulb} converts \textbf{chemical energy} from the battery into \textbf{thermal} and \textbf{light energy}.
    \end{itemize}
\end{itemize}

\section*{Electric Potential (Voltage) Analogy}
\begin{itemize}
    \item \textbf{Electric Potential} is similar to height on a contour map:
    \begin{itemize}
        \item \textbf{Electric field} is analogous to the slope of a hill.
        \item \textbf{Voltage} drives current, similar to how gravitational potential drives water flow.
    \end{itemize}
\end{itemize}

\section*{Kirchhoff’s Current Law (Current Node Rule)}
\begin{itemize}
    \item \textbf{Current Node Rule}: Current entering a node equals current exiting.
    \item Example: In a parallel circuit, current splits among branches but remains conserved.
\end{itemize}

\section*{Electric Field in Wires}
\begin{itemize}
    \item \textbf{Electric field (E-field)} is generated by \textbf{surface charges} on the wires.
    \item Constant current implies a constant E-field throughout the wire.
    \item \textbf{Drude’s Model} explains that electrons lose energy through collisions with lattice defects but continue moving due to the E-field.
    \item \textbf{Mobility} \( (u) \) is a material property controlling drift velocity.
\end{itemize}

\section*{Surface Charges and Circuit Behavior}
\begin{itemize}
    \item The \textbf{surface charge} distribution along the wire creates the E-field.
    \item \textbf{Transient State} occurs when a circuit is first connected, leading to a disturbance in the E-field before reaching steady state.
    \item Surface charges adjust rapidly, restoring equilibrium and maintaining a steady E-field throughout the circuit.
\end{itemize}

\end{document}
